% Présentation générale, technologie front back autres,
% README, markdown, Console
% https://en.wikipedia.org/wiki/Software_design#Design_considerations


\section{Introduction}
\label{sec:introduction}

\begin{frame}
    \frametitle{L'objectif du module}

    \begin{figure}
        \centering
        \includegraphics[width=\linewidth]{figures/introduction/overview}
        \label{fig:overview}
    \end{figure}
\end{frame}

\begin{frame}
    \frametitle{Le Web}

    \begin{figure}
        \centering
        \includegraphics[height=0.5\linewidth]{figures/introduction/http-layers}
        \label{fig:web}
    \end{figure}
\end{frame}

\begin{frame}
    \frametitle{Le protocole HTTP}

    \begin{columns}
        \begin{column}{0.5\textwidth}
            \lstinputlisting[
                language=http,
                caption=Requête,
                label=lst:http-request]
            {figures/introduction/http-request.txt}
        \end{column}

        \begin{column}{0.5\textwidth}
            \lstinputlisting[
                language=http,
                caption=Réponse,
                label=lst:http-response]
            {figures/introduction/http-response.txt}
        \end{column}
    \end{columns}
\end{frame}

\begin{frame}
    \frametitle{Méthodes de requête HTTP}

    \begin{columns}
        \begin{column}{0.8\textwidth}
            HTTP définit un ensemble de méthodes (verbes) de requête qui indiquent l'action que l'on souhaite réaliser sur la ressource indiquée.
        \end{column}
        \begin{column}{0.2\textwidth}

            \qrcode[height=60pt]{https://developer.mozilla.org/fr/docs/Web/HTTP/Methods}
        \end{column}
    \end{columns}

    \begin{itemize}
        \item \textbf{GET} : demande une représentation de la ressource spécifiée, les requêtes GET doivent uniquement être utilisées afin de récupérer des données.
        \item \textbf{POST} : est utilisée pour envoyer une entité vers la ressource indiquée, cela entraîne généralement un changement d'état ou des effets de bord sur le serveur.
        \item \textbf{PUT} : remplace toutes les représentations actuelles de la ressource visée par le contenu de la requête.
        \item \textbf{DELETE} : supprime la ressource indiquée.
    \end{itemize}
\end{frame}


\begin{frame}
    \frametitle{Mon parcours}

    \begin{columns}
        \begin{column}{0.6\textwidth}
            \begin{itemize}
                \item 2020 : Doctorat sur la maintenabilité logicielle
                \item 2017 : Développeur \emph{Java}
                \item 2016 : Expert automatisation \emph{Cucumber}
                \item 2011 : Chef de projet testing
                \item 2008 : Testeur fonctionnel
                \item 2007 : Diplômé de l'ENSC
                \item 2004 : Licence d'électronique, électrotechnique et automatique
            \end{itemize}
        \end{column}

        \begin{column}{0.4\textwidth}
            \begin{figure}
                \centering
                \includesvg[width=0.8\textwidth]{figures/introduction/code-graph}
                \label{fig:code-graph}
            \end{figure}
        \end{column}
    \end{columns}
\end{frame}

\begin{frame}
    \frametitle{Architecture}

    \begin{figure}
        \centering
        \includegraphics[height=0.5\linewidth]{figures/introduction/architecture}
        \label{fig:architexture}
    \end{figure}
\end{frame}

\begin{frame}
    \frametitle{Controlleur}

    \lstinputlisting[
        language=c,
        label=lst:controller]
    {figures/introduction/controller.cs}
\end{frame}


\begin{frame}
    \frametitle{DbContext}

    \lstinputlisting[
        language=c,
        label=lst:dbcontext]
    {figures/introduction/dbcontext.cs}
\end{frame}

\begin{frame}
    \frametitle{Les facettes du génie logiciel}

    \begin{itemize}
        \item l'analyse fonctionnelle
        \item l'architecture
        \item la programmation
        \item les tests
        \item la maintenance
        \item la gestion de projet
        \item l'administration système
    \end{itemize}
\end{frame}

\begin{frame}
    \frametitle{Modélisation UML}

    \begin{figure}
        \centering
        \includegraphics[height=0.5\linewidth]{figures/introduction/uml}
        \label{fig:uml}
    \end{figure}
\end{frame}

\begin{frame}
    \frametitle{L'environnement du développeur}

    \begin{figure}
        \centering
        \includegraphics[width=\linewidth]{figures/introduction/intellij}
        \label{fig:environnement}
    \end{figure}
\end{frame}

\begin{frame}
    \frametitle{Les pratiques du développeur}

    \begin{figure}
        \centering
        \includegraphics[height=0.5\linewidth]{figures/introduction/solid}
        \label{fig:pratiques}
    \end{figure}
\end{frame}

\begin{frame}
    \frametitle{Les base de données}

    \begin{figure}
        \centering
        \includegraphics[height=0.5\linewidth]{figures/introduction/database}
        \label{fig:database}
    \end{figure}
\end{frame}

\begin{frame}
    \frametitle{La collaboration dans une équipe}

    \begin{figure}
        \centering
        \includegraphics[height=0.5\linewidth]{figures/introduction/agile}
        \label{fig:collaboration}
    \end{figure}
\end{frame}

\begin{frame}
    \frametitle{Les tests composants}

    \begin{figure}
        \centering
        \includegraphics[height=0.5\linewidth]{figures/introduction/tests}
        \label{fig:tests}
    \end{figure}
\end{frame}

\begin{frame}
    \frametitle{La contenerisation}

    \begin{figure}
        \centering
        \includegraphics[height=0.5\linewidth]{figures/introduction/docker}
        \label{fig:conteneur}
    \end{figure}
\end{frame}

\begin{frame}
    \frametitle{Les patrons logiciels}

    \begin{figure}
        \centering
        \includegraphics[height=0.5\linewidth]{figures/introduction/patrons}
        \label{fig:patrons}
    \end{figure}
\end{frame}

\begin{frame}
    \frametitle{Le Document d'Architecture Technique}

    \begin{figure}
        \centering
        \includegraphics[height=0.5\linewidth]{figures/introduction/c4-overview}
        \label{fig:dat}
    \end{figure}
\end{frame}

\begin{frame}
    \frametitle{La méthode du canard en plastique}

    \begin{figure}
        \centering
        \includegraphics[height=0.5\linewidth]{figures/introduction/canard}
        \label{fig:canard}
    \end{figure}
\end{frame}
